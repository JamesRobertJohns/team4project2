\documentclass{beamer}
\usetheme{CambridgeUS}
\title{SC2001 Project 2}
\subtitle{The Dijkstra's Algorithm}
\author[Hong, Dinh, He]{Hong Jia Yang \and Dinh Pham Minh Anh \and He Qi Xin}
\institute{Team 4}
\date{\today}

\begin{document}

\begin{frame}
	\titlepage
\end{frame}

\begin{frame}
	\frametitle{Outline}
	\tableofcontents
\end{frame}

\section{Implementation of Dijkstra's}
\subsection{Adjacency Matrix and Array}

\begin{frame}
	\frametitle{Matrix Representation of Graph}
	For a graph \( G = (V, E) \), assume that the vertices are numbered \( 1, 2, \hdots, \lvert{ V }\rvert  \) in some arbitrary manner. Then the adjacency-matrix representation of a graph G consists of a \( \lvert{ V }\rvert \times \lvert{ V }\rvert  \) matrix \( A = (a_{ij}) \) such that
	\[
		a_{ij} = \begin{cases}
			w_{ij} & \text{if } (i, j) \in E,\\
			\infty & \text{otherwise.} 
		\end{cases}
	\]

	Where \( w_{ij} \) is the weight for the weight function \( w(u, v) : E \rightarrow \mathbb{R}^+ \) 
	\onslide<2> \begin{block}{Recall}
		Dijkstra's only works for non-negative weighted Directed Graphs	
	\end{block}
\end{frame}

\begin{frame}
	\begin{columns}
	\column{0.5\textwidth}
	\begin{figure}
		\includegraphics[scale=0.5]{./pict/exampleGraph.png}
		\caption{Graphical Example}
	\end{figure}

	\column{0.5\textwidth}
	\begin{table}
		\begin{tabular}{c|c|c|c|c|c|c}
			  & 1 & 2 & 3 & 4 & 5 & 6\\
			  \hline
			1 & $\infty$ & 1 & $\infty$ & $\infty$ & 1 & 1\\
			\hline
			2 & $\infty$ & $\infty$ & $\infty$ & $\infty$ & 1 & 1\\ 
			\hline
			3 & $\infty$ & $\infty$ & $\infty$ & 1 & 1 & $\infty$ \\
			\hline
			4 & $\infty$ & $\infty$ & $\infty$ & $\infty$ & $\infty$ & $\infty$  \\
			\hline
			5 &  $\infty$ & $\infty$ & $\infty$ & $\infty$ & $\infty$ & 1\\
			\hline
			6 & $\infty$ & $\infty$ & $\infty$ & $\infty$ & $\infty$ & $\infty$\\
			\hline
		\end{tabular}
		\caption{Matrix Representation}
	\end{table}
	\end{columns}
\end{frame}

\subsection{Adjacency List and Minimising Heap}
\begin{frame}
	\begin{columns}
	\column{0.5\textwidth}
	\begin{figure}
		\includegraphics[scale=0.5]{./pict/exampleGraph.png}
		\caption{Graphical Example}
	\end{figure}

	\column{0.5\textwidth}
	\begin{align*}
		1 &: (3, 1) \rightarrow (5, 1) \rightarrow (6, 1)\\
		2 &: (5, 1) \rightarrow (6, 1) \\
		3 &: (4, 1) \rightarrow (5, 1) \\
		4 &: \\
		5 &: (6, 1)\\
		6 &: 
	\end{align*}			
	\centering List Representation
		
	\end{columns}
\end{frame}


\section{Complexity Analysis and Comparison}
\begin{frame}
	
\end{frame}

\end{document}
